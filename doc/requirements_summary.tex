% Generated by GrindEQ Word-to-LaTeX 2010 
% ========== UNREGISTERED! ========== Please register! ==========
% LaTeX/AMS-LaTeX

\documentclass{article}

%%% remove comment delimiter ('%') and specify encoding parameter if required,
%%% see TeX documentation for additional info (cp1252-Western,cp1251-Cyrillic)
%\usepackage[cp1252]{inputenc}

%%% remove comment delimiter ('%') and select language if required
%\usepackage[english,spanish]{babel}

\usepackage{amssymb}
\usepackage{amsmath}
\usepackage[dvips]{graphicx}
%%% remove comment delimiter ('%') and specify parameters if required
%\usepackage[dvips]{graphics}

\begin{document}

%%% remove comment delimiter ('%') and select language if required
%\selectlanguage{spanish} 

\noindent \textbf{Current Idaho TARS Requirements}

\noindent 

\noindent \textbf{PCA -1}

\noindent Users with the proper permissions must be able to manually enter PCA codes in a form that meets DHW standards.

\noindent Status: \textbf{In Progress.  All of the core functionality exists to add, edit, and delete PCA codes to and from the database.  However, there is no user friendly interface.}

\noindent \textbf{}

\noindent \textbf{PCA-2}

\noindent Users with proper permissions must be able to manually tie work effort(s) to valid PCA.

\noindent \textit{This provides the ability to correctly connect the PCA code(s) to work, and ensures we are correctly accounting for what costs are allocated to what work. Work can be project related, a maintenance activity, or a non-productive activity such as meeting time.}

Status: \textbf{In Progress.  All of the core functionality exists to create associations between PCA codes and Work Efforts.  However, there is no user friendly interface.}



\noindent \textbf{PCA-3}

\noindent The system must provide a mechanism for time bounding PCA codes - with the ability to "deactivate" a code prematurely and an open "end" date.

\noindent \textit{Codes need to have start and end dates assigned to them, and depending upon user permissions, those dates may or may not be editable. Do not mimic MS Projects start and end date functionality.}

\noindent Status: \textbf{In Progress.  There exists a flag to indicate that a PCA code is active or not.  However, there exists no user interface to manually deactivate PCA codes nor is there any automated system to check PCA code's time bounds.}



\noindent \textbf{PCA-4}

\noindent Must maintain an audit trail (history of changes - people, projects, and PCAs).

\noindent \textit{Maintain history - history must remain static, dates, time (hours) cannot change once booked (accounting term for in system).}

\noindent Status: \textbf{Complete.  All user initiated database interactions cause an entry into the History table.}

\noindent 

\noindent \textbf{PCA-5}

\noindent The system must provide a mechanism for preventing time to be allocated to expired PCA codes.

\textit{No Comments}

\noindent Status: \textbf{In Progress.  The flag exists but as of yet no check is made because no permanent user interface yet exists.}

\noindent 

\noindent \textbf{PCA-6}

\noindent The system must allow multiple PCA to be assigned to a work effort, over the life of the effort/project.

\textit{No Comments}

\noindent Status: \textbf{In Progress.  The functionality exists to allow multiple associations as well as the removal of associations but there is no visual interface for it yet.}

\noindent 

\noindent \textbf{PCA-7}

\noindent Must be able to assign one or more PCA codes to work effort (split \% allocation across multiple PCAs which can change during life of work effort).

\noindent \textit{In the case where a work effort is tied to more than one PCA, there needs to be some mechanism for determining how the work is partitioned between the two.}

\noindent Status: \textbf{In Progress.  Some functionality exists but is not complete and no visual interface exists.}

\noindent 

\noindent \textbf{PCA-8}

\noindent Must allow work to be assigned to other entities outside DHW.

Provides a means of identifying between DHW and non-DHW contractor hours.

\noindent Status: \textbf{In Progress.   Work efforts can be associated to a user but there currently exists no visual way to manually associate work to someone other than the user currently logged in.}



\noindent \textbf{PCA-9}

\noindent Must allow work to be associated with multiple divisions or the enterprise.

\textit{No Comments}

 Status: \textbf{Incomplete.}

\noindent 

\noindent \textbf{DATA-1}

\noindent The system shall track date specific vendor and employee/contractor information.

\noindent \textit{Provide some mechanism for tracking employment status and changes. For example, when a contractor is hired on as state staff, or changes vendors, TARS would allow that information to be entered and tracked. Need data elements.}

\noindent Status: \textbf{In Progress?  Employee Data is stored in the Active Directory.  If something separate is needed than it will need to be added as there is currently no functionality for the storage of data not associated to PCA Codes, Work Efforts, or Tasks.}\textit{}

\noindent \textit{}

\noindent \textbf{DATA-2}

\noindent The system shall allow for some description of work or project to be entered and attached.

\textit{Provide ability to describe the work effort in general terms.}

Status: \textbf{Complete.  There exists a field for comments to be added to hours logged.}\textit{}

\noindent 

\noindent \textbf{DATA-3}

\noindent The system shall be consistent with I-Time data.

\textit{Use the codes and logic from I-Time (see DAT-7 and REP-2) codes are Earning Codes - 3 digit}

\noindent Status: \textbf{Tabled.  This aspect of the project was declared to be outside of the scope of this semester and was tabled.}\textit{}

\noindent 

\noindent \textbf{DATA-4}

\noindent The system shall provide a means to replicate last week's assignments (repeating tasks can auto fill)

\noindent \textit{For those staff/contractors that repeat most work efforts each week, having the ability to replicate the proceeding week saves data entry time.}

\noindent Status: \textbf{In Progress.  Some functionality exists but a controller will need to be added to pull the last approved entry from the database upon request by the user.  Also a view will be needed.}\textit{}

\noindent 

\noindent \textbf{DATA-5}

\noindent The system must have a method that allows staff to create work effort, and self-assign

\noindent \textit{Users would have the ability to create work effort, then assign themselves to that effort.} \textit{Clarification - Staff can self assign to a particular work effort, Mgrs and Admin only can CREATE a work effort with a PCA}

Status: \textbf{Complete.}\textit{}

\noindent 

\noindent \textbf{DATA-6}

\noindent Must be able to track work effort for resources, depending upon their assignment, that are either cost allocated or not cost allocated. 

\noindent Clarification - Cost allocated codes are PCA codes (all work is entered unto I-time as ACT, all non-work time captures is by Earnings Codes\textit{}

Status: \textbf{In Progress.  The functionality exists but there are no visual elements yet.}\textit{}

\noindent 

\noindent \textbf{DATA-7}

\noindent Must be able to break time out by time codes for work efforts, such as Vacation, Sick, LWOP, (match I-Time data since this is the system of record)

\noindent \textit{Need to understand data requirement for I-Time if the plan is to eventually interface with I-Time.} \textit{All time in I-time is seen as worked or non-worked.  All "work" is coded to ACT, non work is coded to various codes.}

\textit{ }Status: \textbf{In Progress.  The functionality exists but there are no visual elements yet.}

\noindent 

\noindent \textbf{DATA-8}

\noindent Users shall have the ability to close tasks and activities on their timesheet, and reopen if needed.

\noindent \textit{This is separate from the open/closed PCA codes.  A user may no longer be working on a particular project or investment so they want to close it out on their time sheet.}

\noindent Status: \textbf{Progress.  This level of visual design was beyond the scope of this semester.  However, some functionality exists.}

\noindent 

\noindent \textbf{DATA-9}

\noindent The system shall provide some mechanism (configurable dropdown) for grouping of business, program, and function of work.

Need to be able to add/edit/delete values into those lists. This is grouping for a work effort

\noindent Status: \textbf{Progress.  This level of visual design was beyond the scope of this semester.  However, some functionality exists.}

\noindent 

\noindent \textbf{DATA-10}

\noindent Audit trail data shall include the information that was updated, modified/deleted, date created, and by whom for each item determined to be auditable. The TARS team has proposed storing every SQL query executed by a user.

 \textit{Need to define what will be included in the audit trail.}

\noindent Status: \textbf{Complete.  The History table exists and is populated as the system is used but no visual interface exists to view said histories.}\textit{}

\noindent 

\noindent \textbf{DATA-11}

\noindent Data for staff and projects shall include the ability to store links and attachments

\noindent Related in part to DAT-2, in that it provides a means for capturing work and project information that describes the work effort.

\noindent Status: \textbf{In Progress.  Some functionality exists but nothing has been done yet concerning storing files} 

\noindent 

\noindent \textbf{}

\noindent \textbf{}

\noindent \textbf{}

\noindent \textbf{}

\noindent \textbf{DATA-12}

\noindent The system must allow for future time entry

Any user should be allowed to enter time against a work effort, in advance of the current week.

Status: \textbf{Complete.  No restriction is made on how far ahead time can be entered for.  Some  might be warranted.}

\noindent 

\noindent \textbf{DATA-13}

\noindent Must prevent work efforts to exist in the system unless they are tied to a PCA code.

\noindent PCA codes and work efforts (tasks, \dots ) are all time bounded in this system. To prevent inaccurate recording of time allocated to an effort, some automated process of preventing expired or deactivated objects should be developed. System must also allow non-work time to be recorded using Earnings Codes (I-Time) such as VAC, HOL, etc.

\noindent Status: \textbf{In Progress.  The functionality to make associations exists but not to require them.}

\noindent 

\noindent \textbf{REP-1}

\noindent All data for reporting shall be extracted via external source (EDW. Excel, etc.).

\noindent Team sees no need to build in reporting in TARS, since we can generate reports with Business Objects, or other database connections. 

\noindent Status: \textbf{Inactive.   This was declared out of the scope of this semester and has not been considered at all yet.}

\noindent 

\noindent \textbf{REP-2}

\noindent Must allow users to create a view of their I-Time timesheet.

I-time is a separate timesheet, into which users also enter time for payroll accounting. (This requirement will be prioritized at the very bottom.  Isn't needed until we interface with I-Time)

\noindent Status: \textbf{Inactive.   This was declared out of the scope of this semester and has not been considered at all yet.}

  

\noindent \textbf{REP-3}

\noindent Reports must be real-time, reliable, and accurate. Includes exports to csv, Excel.

\noindent Speaks to having a simplified database schema, one that allows external connections (ODBC, etc.) to easily connect and extract data for reporting purposes. Reports will be created using Business Objects, not in TARS

\noindent Status: \textbf{Inactive.   This was declared out of the scope of this semester and has not been considered at all yet.}

\noindent 

\noindent \textbf{VIEW-1}

\noindent Must have a sort and group function that allows work effort to be grouped by application, division, manager, etc.

 \textit{No Comments}

\noindent Status: \textbf{In Progress.  The only views currently are for testing purposes.  Thus all the ``View'' requirements are In Progress.  Visual Design will be a primary focus of the second semester.}\textit{ }

\noindent \textit{}

\noindent \textbf{VIEW-2}

\noindent The system must allow a user the ability to create a custom view of the data.

\noindent Users should be able to slice data, such as work effort by staff member over a date range. Users should only be able to see and customize their own data unless they are mgrs or admin and the view should persist.

Status: \textbf{See View-1}

\noindent 

\noindent \textbf{VIEW-3}

\noindent Must allow users to easily size windows

 \textit{No Comments}

\textit{ }Status: \textbf{See View-1}\textit{}

\noindent 

\noindent \textbf{VIEW-4}

\noindent Must be able to limit view of information presented to user to what is pertinant to that user's role.

This requirement is tied to VEW-2 in that it limits the range of customization of a view.

Status: \textbf{See View-1}

\noindent 

\noindent \textbf{VIEW-5}

\noindent The system shall provide search/find functionality to locate work efforts, with minimal amount of navigation (task actions$<$=4 clicks/pages/dialogs)

\noindent The number of clicks should be proportional to the frequency of the TARS tasks. In other words, TARS tasks that users frequently execute, should have the fewest navigation steps. Users can only create a view of their own data. Mgrs should be able to group their staff for time approval.

\noindent Status: \textbf{See View-1}

\noindent 

\noindent \textbf{SEC-1}

\noindent Must authenticate using LDAP.

\noindent \textit{Initially authenticating to Active Directory was a requirement. For now, we are using Apache DS to authenticate users. }

\noindent Status: \textbf{In Progress.  Currently authenticates using LDAP.  Will need to be modified to authenticate to Active Directory.}\textit{}

\noindent \textit{}

\noindent \textbf{SEC-2}

\noindent Must have a role-based permissions security.

\noindent Would like to have the ability to create new roles, and assign permissions to that role. For example, an Administrator have rights to edit/delete PCA codes and users, while an Individual Contributor would not have those rights. Basic role set would include Administrator, Manager /Approver, and Worker.

\noindent Status: \textbf{Inactive.  This is taken care of by the Active Directory system and thus has been striken from this list.}

\noindent 

\noindent \textbf{SEC-3}

\noindent The system shall allow for automated closure of time periods for PCA and work efforts, with administrator ability to manually reopen \& close for edit \& approval

\noindent Tied to DAT-13 in preventing work efforts and codes from lingering when they are no longer active. This is also an example of a permission element in the role-based security profile.

\noindent Status: \textbf{In Progress.  The flag to mark a PCA code and Work Effort as closed and the field to set the time bounds exist.  However, no automated system to set that flag when the time runs out has been implemented.}

\noindent 

\noindent \textbf{NAV-1}

\noindent The system must allow each user the ability to navigate easily by logic/functional areas, ie. Staff demographics, projects, work items/areas, time entry, etc.

 \textit{No Comments}

Status: \textbf{See View-1}\textit{}

\noindent 

\noindent \textbf{NAV-2}

\noindent Must automatically display current week when entering timesheet data.

 \textit{No Comments }

\textit{ }Status: \textbf{See View-1}\textit{}

\noindent 

\noindent \textbf{WKF-1}

\noindent Must have notifications (via email, context, \dots ) triggered by certain events such as timesheet submittal, approvals, PCA expiration, etc.

\noindent The current system auto-sends emails to notify users of their due timesheets, though the message is not tied to timesheet status (i.e. you get the mail even if your timesheet was submitted for that week). One notification would be to the worker who has NOT submittee his/her time by end of day on Saturday.

\noindent Status: \textbf{Inactive.  This was declared out of the scope of this semester and was not addressed.}

\noindent 

\noindent \textbf{WKF-2}

\noindent Users with permissions, must have the ability to approve TARS weekly submittals

\noindent Assumes ability to view other's timesheets based on your role permission. Rejected timesheets will provide notification to the submitter.

\noindent Status: \textbf{In Progress.  The field to flag a timesheet as approved exists but there is no visual way to set that flag.}

\noindent 

\noindent 

\noindent \textbf{Idaho TARS Requirements ChangeLog}

\noindent 
\[10/2/2011\] 

 SEC-2 Moved to \textbf{Inactive. }Requirement satisfied in SEC-1.

\noindent 
\[11/2/2011\] 

 \textbf{Clarified existing requirements for:}

\textbf{  DAT-1}

\textbf{  DAT-2}

\textbf{  DAT-3}

\noindent \textbf{DAT-5}

\noindent \textbf{DAT-6}

\noindent \textbf{DAT-7}

\noindent \textbf{DAT-8}

\noindent \textbf{DAT-9}

\noindent \textbf{DAT-10}

\noindent \textbf{DAT-13}

\noindent \textbf{REP-1}

\noindent \textbf{REP-2}

\noindent \textbf{VIEW-2}

\noindent \textbf{VIEW-5}

\noindent \textbf{SEC-2}

\noindent \textbf{WKF-1}

  

 REP-3 Moved to \textbf{Inactive. }Reports will be generated using business objects, not TARS.

\noindent 
\[11/8/2011\] 

 Updated SEC-1. 

\textbf{ }


\end{document}

% == UNREGISTERED! == GrindEQ Word-to-LaTeX 2010 ==

